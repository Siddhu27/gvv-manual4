\begin{enumerate}[label=\thesection.\arabic*.,ref=\thesection.\theenumi]
\numberwithin{equation}{enumi}
\item State the general model of a state space system specifying the dimensions of the matrices and vectors.
\\
\solution 
\begin{align*}
\dot{\vec{x}(t)}&=\vec{A}\vec{x}(t)+\vec{B}\vec{u}(t) \\
 \vec{y}(t)&=\vec{C}\vec{x}(t)+D u(t)
\end{align*}
\\ x(.) is called the "state vector"
\\ y(.) is called the "output vector"
\\ u(.) is called the "input(control) vector"
\\ A(.) is the "state or system matrix", dim[A(.)]=nxn
\\ B(.) is the "input matrix",dim[B(.)]=nxp
\\ C(.) is the "output matrix",dim[C(.)]=qxn
\\ D(.) is the "feedthrough matrix",dim[A(.)]=qxp
\\ $\dot{x(t)}=\frac{d}{dt}x(t)$
\\
\item Find the transfer function $\vec{H}(s)$ for the general system.
\solution
\\ \textbf{FINDING TRANSFER FUNCTION}
\begin{align*}
 \dot{X(t)}=AX(t)+BU(t) \\
 Y(t)=CX(t)+DU(t)
\end{align*} 
\\
\\   
\\by applying laplace transforms on both sides of equation 1
we get
\begin{align*}
s.X(s)-X(0)=A.X(s)+B.U(s) \\
s.X(s)-A.X(s)=B.U(s)+X(0) \\
(sI-A)X(s)=X(0)+B.U(s) \\
X(s)=X(0)([sI-A])^{-1}+(([sI-A])^{-1}B)U(s)
\end{align*}
\\Laplace transform of equation 2 and sub X(s) 
\begin{align*}
Y(s)=C.X(s)+D.U(s)
\end{align*}
\begin{align*}
Y(s)=C.[X(0)([sI-A])^{-1 }+ (([sI-A])^{-1}B)U(s)]+DU(s) 
\end{align*}
\\If X(0)=0
\begin{align*}
Y(s)=C[(([sI-A])^{-1}B)U(s)]+DU(s) \\
\frac{Y(s)}{U(s)}=C[(([sI-A])^{-1}B)]+D=H(s) 
\end{align*}
\item Given 
\begin{align*}
H(s)=\frac{1}{s^3+3s^2+2s+1}
\end{align*}
\begin{equation*}
 b1 =\begin{vmatrix}
  0&0&1\\
 \end{vmatrix}
\end{equation*}
\begin{align*}
b1^T=B
\end{align*}
and D=0, find A and C.
\\
\solution
\\ In the generic form D is feedforward matrix of dimension qxp
\\ where q is number of outputs and  p is number of inputs
\\ from given question as D is given 0(scalar), so q=p=1
\\ therefore it is single input and single output system
\\ Hence U is the input and Y is the output which are scalar
\\ Given that B is 3x1 so, n=3
\\ so, A is 3x3,C is 1x3
\\ As we know that
\begin{align*}
Y(s)=H(s) \times U(s)= (\frac{1}{s^3+3s^2+2s+1}) \times U(s) \\ \\
H(s)=\frac{Y(s)}{U(s)}=(\frac{(\frac{Y(s)}{x_{1}(s)})}{\frac{U(s)}{x_{1}(s)}})=\frac{1}{s^3+3s^2+2s+1} 
\end{align*}
\\let 
\begin{align*}  
x_{1}(s)=\frac{U(s)}{s^3+3s^2+2s+1} \\ \\
Y(s)=x_{1}(s)\times 1 
\end{align*}
\begin{align}
s^3x_{1}(s)+3s^2x_{1}(s)+2sx_{1}(s)+x_{1}(s)=U(s)
\end{align}
\\ Taking inverse laplace transform we get
\begin{align*}
\dddot{x_{1}(t)}+\ddot{x_{1}(t)}+\dot{x_{1}(t)}+x_{1}(t)=U(t)
\end{align*}
\begin{align*}
\dot{x_{1}}=x_{2} \\ 
\ddot{x_{1}}=\dot{x_{2}}=x_{3} \\ 
\dddot{x_{1}}=\ddot{x_{2}}=\dot{x_{3}}
\end{align*}
\\ so equation 1.3.1 can be written as
\\
\begin{gather*}
\begin{bmatrix}
sx_{1}(s)\\
s^2x_{1}(s)\\
s^3x_{1}(s)
\end{bmatrix}
=
\begin{bmatrix}
0&1&0\\
0&0&1\\
-1&-2&-3
\end{bmatrix}\times \begin{bmatrix}
x_{1}(s)\\
sx_{1}(s)\\
s^2x{1}(s)
\end{bmatrix}
+
\begin{bmatrix}
0\\
0\\
1
\end{bmatrix} \times U
\end{gather*}
taking inverse laplace transform
\begin{gather*}
\begin{bmatrix}
\dot{x_{1}}\\
\dot{x_{2}}\\
\dot{x_{3}}
\end{bmatrix}
=
\begin{bmatrix}
0&1&0\\
0&0&1\\
-1&-2&-3
\end{bmatrix}\times \begin{bmatrix}
x_{1}\\
x_{2}\\
x_{3}
\end{bmatrix}
+
\begin{bmatrix}
0\\
0\\
1
\end{bmatrix} \times U
\end{gather*}
therfore
\begin{equation*}
A=\begin{bmatrix}
0&1&0\\
0&0&1\\
-1&-2&-3
\end{bmatrix}
\end{equation*}
\\
Since $ Y(s)=x_{1}(s)\times numerator$
\\therefore $ Y(s)=x_{1}(s) $
\begin{gather*}
\\Y=
\begin{bmatrix}
1&0&0
\end{bmatrix}\times \begin{bmatrix}
x_{1}(s)\\
sx_{1}(s)\\
s^2x_{1}(s)
\end{bmatrix} 
\end{gather*}
taking inverse laplace transform
\begin{gather*}
\\Y=
\begin{bmatrix}
1&0&0
\end{bmatrix}\times \begin{bmatrix}
x_{1}\\
x_{2}\\
x_{3}
\end{bmatrix} 
\end{gather*}
\begin{equation*}
C=\begin{bmatrix}
1&0&0
\end{bmatrix}
\end{equation*}
%\end{document}

%\end{document}
\end{enumerate}

